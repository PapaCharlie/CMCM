\documentclass[titlepage]{article}
% \usepackage[margin=2.5cm]{geometry}
\usepackage[margin=2.5cm, headheight=0pt, headsep=1cm]{geometry}
\usepackage{enumerate, fancyhdr, graphicx, amsmath, nth}
\usepackage[binary-units=true]{siunitx}

\title{Hurricane Evacuation Strategies for the State of Mississippi}
\author{Paul Chesnais (pmc85), Antoine Pourchet (app63), and Ryan Vogan (rcv39)\\2015 Cornell Mathematical Contest in Modeling}
\date{November 16, 2015}

\pagestyle{fancy}
\fancyhead{}
\lhead{Chesnais, Pourchet, Vogan}
\chead{Hurricane Evacuation Strategies}
\rhead{November 16, 2015}
\fancyfoot{}
\rfoot{\thepage}
\renewcommand{\headrulewidth}{0.5pt}
\renewcommand{\footrulewidth}{0.5pt}

\usepackage{listings, color, times, textcomp, float, hyperref, setspace, subcaption}
\definecolor{Code}{rgb}{0,0,0}
\definecolor{Decorators}{rgb}{0.5,0.5,0.5}
\definecolor{Numbers}{rgb}{0.5,0,0}
\definecolor{MatchingBrackets}{rgb}{0.25,0.5,0.5}
\definecolor{Keywords}{rgb}{0,0,1}
\definecolor{self}{rgb}{0,0,0}
\definecolor{Strings}{rgb}{0,0.63,0}
\definecolor{Comments}{rgb}{0,0.63,1}
\definecolor{Backquotes}{rgb}{0,0,0}
\definecolor{Classname}{rgb}{0,0,0}
\definecolor{FunctionName}{rgb}{0,0,0}
\definecolor{Operators}{rgb}{0,0,0}
\definecolor{Background}{rgb}{0.98,0.98,0.98}
\definecolor{mygreen}{RGB}{28,172,0}
\definecolor{mylilas}{RGB}{170,55,241}

\lstdefinestyle{Python}{
  backgroundcolor=\color{Background},basicstyle=\ttfamily\small\setstretch{1},breaklines=true,
  commentstyle=\color{Comments}\slshape,emph={self},emphstyle={\color{self}\slshape},frame=l,framexbottommargin=2em,
  framextopmargin=2em,keywordstyle={[2]\color{Decorators}\slshape},keywordstyle={\color{Keywords}\bfseries},
  language=Python,morecomment=[s][\color{Strings}]{"""}{"""},morecomment=[s][\color{Strings}]{'''}{'''},
  morekeywords={[2]@invariant},morekeywords={import,from,class,def,for,while,if,is,in,elif,else,not,and,or,print,break,
  continue,return,True,False,None,access,as,del,except,exec,finally,global,import,lambda,pass,print,raise,try,assert},
  numbers=left,numbersep=1em,numberstyle=\footnotesize,showspaces=false,showstringspaces=false,showtabs=false,
  stringstyle=\color{Strings},tabsize=4,xleftmargin=1em,
}

\lstdefinestyle{Matlab}{
  backgroundcolor=\color{Background},basicstyle=\ttfamily\small\setstretch{1},breaklines=true,
  commentstyle=\color{Comments}\slshape,emph=[1]{for,end,break},emphstyle=[1]\color{red},frame=l,identifierstyle=\color{black},
  keywordstyle=[2]{\color{black}},keywordstyle={\color{blue}\bfseries},language=Matlab,literate={~} {\texttildelow}{1},
  morekeywords=[2]{1},morekeywords={matlab2tikz},numbers=left,numbersep=1em,numberstyle=\footnotesize,
  showstringspaces=false,stringstyle=\color{Strings},xleftmargin=1em,
}

\begin{document}
\maketitle
\thispagestyle{empty}

\section{Executive Summary}
\label{sec:summary}
\vspace{1cm}
Governor Bryant,\\

At the request of the Mississippi Emergency Management Administration, we have conducted a thorough mathematical analysis of hurricane evacuation strategies for all 82 counties in the state of Mississippi. These strategies were tested across a range of possible hurricane conditions backed by historical weather data. By testing across this range of possibilities, we found that it is best to follow one of two different strategies depending on the type of storm being faced.\\

In the event of a moderately sized, category 1, 2, or 3 hurricane, we recommend a strategy of evacuating every county in the coastal region. These evacuations should head in a direction that is most pointed away from the expected landfall. If many of the direction choices are very similar, these counties should prefer their north-west neighbors, as hurricanes from the gulf tend to curve toward the north-east. The virtues of this strategy are that it's relatively localized and allows each county to independently pick it's evacuation direction. As a result of this, evacuations of this type are very quick to organize, and allow for a rapid evacuation of the most endangered areas. This is useful for moderately sized hurricanes because it allows you to take more time in deciding whether or not to evacuate. You can wait to see if the storm will die down before hitting the coast, but are still able to spring into action if it doesn't.\\

In the event of a very large, category 4 or 5 hurricane, we recommend an orderly evacuation of the entire lower half of Mississippi toward the northern border. If a storm of this type appears on the radar, it is very unlikely to die down before hitting the coast, and is more likely to pass over Mississippi with its increased size. Strong governmental organization is required as early as possible to prevent a catastrophic loss of life. We have code capable of solving for the optimal allocation of traffic on each highway to maximize the throughput of the entire evacuation. This is a very heavy-handed response that should be reserved for the most dire of circumstances. In this event that this is needed, we encourage you to consult us for a solution tailored to that specific storm.\\

While these two strategies can aid you in making decisions for an imminent hurricane, we also have a recommendation for a long-term improvement to all evacuation routes. In our analysis of evacuation throughput, we discovered that the area around George County poses a significant bottleneck for highway traffic. As our analysis will show later in this manuscript, adding an additional highway route from George County to Perry county will significantly shift this bottleneck upward. This would allow more evacuees to escape the most dangerous coastal areas before they are slowed down by a bottleneck. Especially with potentially heavy traffic coming up highway 59 from New Orleans, traffic congestion in the coastal region is a problem requiring a more long term solution.\\
\\
\\
Sincerely,\\
Paul Chesnais, Antoine Pourchet, and Ryan Vogan

\newpage
\section{Introduction}
\label{sec:introduction}
  The Mississippi Emergency Management Administration (MSEMA) has hired our team to design hurricane evacuation routes for at-risk counties in the coastal regions of southern Mississippi. Given that this coastal area is a strong commercial center, the government authorities want us to minimize the number of unnecessary evacuations. Under this requirement, it is more advantageous to withhold evacuations for as long as safely possible. The reasons for this are two-fold. First, this ensures that local businesses can remain open for as long as possible. Second, this ensures that evacuation decisions are made when forecasts are most accurate. Therefore, our models seek to recommend an optimal evacuation plan as well as the latest possible time at which it is safe to enact this plan.

\section{Network Representation of Mississippi}
\label{sec:representation}
  \par
    The MSEMA specified that our plan should be enacted on a per-county basis. As you can see in the following figure, these counties are approximately equally sized. As a result of this, we assume that each county can be modeled as a single node in a bidirectional graph. Then an edge from node $i$ to node $j$ represents an available evacuation route from county $i$ to county $j$. In all of our models we assume that highways are the only means of evacuation. Therefore, we constructed a network of Mississippi counties by connecting each county's node to the nodes of all bordering counties directly reachable by highways:
  \begin{figure}[H]
    \center
    \begin{subfigure}[b]{0.5\textwidth}
      \center
      \includegraphics[width=.9\linewidth]{figures/county_map.jpg}
      \caption*{A Map of Counties in the State of Mississippi \cite{county_map}}
    \end{subfigure}~
    \begin{subfigure}[b]{0.5\textwidth}
      \center
      \includegraphics[width=\linewidth]{figures/full_undirected-crop.pdf}
      \caption*{Undirected Graph Representing Mississippi Counties}
    \end{subfigure}
    \caption*{}
  \end{figure}
  \par
     Because Mississippi has varying highway structure, it would be inappropriate to assume uniform travel rates between all county pairs. In order to capture this heterogeneity in the network model, we decided to assign weights to every edge. Then a single edge weight represents a carrying capacity for the maximum amount of travel between two counties. For a real-world mapping of this, we assume that this would correspond to the total number of highway lanes spanning two counties. Therefore, we were able to assign edge weights across the entire graph by using the following highway map:
  \begin{figure}[H]
    \centering
    \includegraphics[width=.5\textwidth]{ms_highways.pdf}
    \caption{Map of Major Highways in the State of Mississippi \cite{ms_highways}}
    \label{fig:highway_map}
  \end{figure}
  \par
    In this figure, the thick black lines represent four lane highways while the red lines represent two lane highways. We assume that despite the ensuing panic, evacuees still travel along the right side of the highway, even if this means lanes going the other way (towards the hurricane) are underutilized. Therefore, for every thick black line spanning two counties, the appropriate edge in the graph gains two additional lanes, whereas red lines add one lane. Summing the lane contributions of all highways for a pair of counties therefore gives us an single edge weight in the graph.
  \newline
  \par
    We assume that there is a fixed spacing between every car on every highway. Then all cars travel at the same speed, and the density of cars is uniform along the entire length of every highway. This determines a fixed flow rate per edge capacity, and allows us to compute the number of cars traveling to another county in a given time interval. Because of the inevitable traffic congestion during an evacuation, we set the fixed speed to a relatively slow $10$ km/hr. In a traffic jam situation like this, we expect a tight traffic density of one car every 10 meters. Therefore:
    \[
      \frac{10\ \text{km/hr}}{10\ \text{m/car}} = 1000\ \text{cars/hr}
    \]
  \par
    We assume that evacuees travel with friends and family in their cars. Because the average size of an American family is \texttildelow 3 persons \cite{famsize}, we conclude that 3000 people per hour can use a single lane to another county. Maximum population flow rates between each county are therefore three thousand times the corresponding edge weight.

\section{Historical Hurricane Trends}
\label{sec:hurricanes}
  Before designing potential evacuation strategies, we first sought to understand the expected speed, size, and landfall of a hurricane impacting Mississippi. Discovering patterns in these storms proved very difficult, because the Earth's atmosphere is a chaotic system. Other than the general pattern that hurricanes in the Gulf of Mexico tend to curl back toward the north-east, there is very little we can predict about a hurricane's trajectory:
   \begin{figure}[H]
    \centering
      \includegraphics[width=.5\textwidth]{figures/chaos.jpg}
      \caption{Map of Atlantic Hurricane Trajectories \cite{chaos}}
    \end{figure}
  As a result of this, we decided to test our evacuation plans against hurricanes that actually occurred in the past. This ensures that our model parameters governing the hurricane are as accurate as possible. In order to do this, we parsed a NOAA dataset containing every cyclone (tropical storm, hurricane, etc.) occurring from 1851 to 2014. For each hurricane in this dataset, we found the first latitude and longitude datapoint over Mississippi and recorded that as a landfall position (throwing out hurricanes that never passed over Mississippi). You can see these points in the figures below, where the left is a web-app display of hurricane trajectories over Mississippi, while the right is the 15 discovered landfalls we pulled from the NOAA dataset:
    \begin{figure}[H]
      \center
      \begin{subfigure}[b]{0.5\textwidth}
        \center
        \includegraphics[width=\linewidth]{figures/smaller_chaos.png}
        \caption*{Map of Mississippi Hurricane Trajectories \cite{smaller_chaos}}
      \end{subfigure}~
      \begin{subfigure}[b]{0.5\textwidth}
        \center
        \includegraphics[width=\linewidth]{figures/parsed_chaos.png}
        \caption*{Landfall Locations Parsed From Hurricane Dataset \cite{fifteen}}
      \end{subfigure}
    \end{figure}
    \newpage
  Each point in this dataset had an associated wind speed that we used to classify it's category \cite{categories}:
  \begin{itemize}
    \item[]
      \textbf{Category 1:} 74-95 mph
    \item[]
      \textbf{Category 2:} 96-110 mph
    \item[]
      \textbf{Category 3:} 111-129 mph
    \item[]
      \textbf{Category 4:} 130-156 mph
    \item[]
      \textbf{Category 5:} 157 mph or higher
  \end{itemize}
  The last piece of data we needed for these hurricanes were their radii. Since hurricane radii records don't exist for years earlier than 2004, we used a least squares linear fit to interpolate the radius for a given hurricane wind speed:
  \begin{figure}[H]
    \centering
    \includegraphics[width=\textwidth]{figures/least_fit.png}
    \caption{Hurricane Radius (nmi) vs Wind Speed (kt)}
    \label{fig:highway_map}
  \end{figure}


\section{Markov Process Analysis of Strictly Northern Evacuations}
\label{sec:markov}
  Because our analysis of historical hurricane trends concluded that the vast majority of landfalls occur in the three southern-most counties, the first evacuation strategy we tested was one in which all evacuees move strictly northward.
  \subsection{Model Description}
    Given that we have a network structure and associated population flow rates, this problem lends itself well to a Markov process analysis. We can then model our evacuation strategy by converting the network of Mississippi counties into a directed graph. For our northward evacuation, this means only drawing highway-directed edges toward nodes with higher latitudes:
    \begin{figure}[H]
      \centering
      \includegraphics[width=.5\textwidth]{figures/full_directed_NS.pdf}
      \caption{Map of Northbound Highway Routes in the State of Mississippi \cite{ms_highways}}
    \end{figure}
    The thickness of each arrow in this graph represents the magnitude of a transition probability from one node to another. In evacuation terms, these are the proportions of a county's population that travel to another county during a single update. The remaining population that does not leave a county during a single step implicitly loops back into the same node. This fulfills the conservation of probability required for all Markov models, preventing population from slowing leaking out of the system. Note that for the counties along the northern border of Mississippi, this necessitates a strictly looping behavior because there are no further northern counties. This results in a key assumption that the evacuated population aggregates in the northernmost counties over time, where they are considered to be completely safe from the hurricane. \\
    \\
    A second key assumption is that the transition probabilities of the graph remain fixed over the entire course of the simulation. The real world implications of this are that county populations converge toward 0 rather than allowing for a full evacuation in a finite number of time steps. We consider this assumption to be reasonable because in a real world scenario it becomes increasingly difficult to evacuate straggling population members who have a reduced mobility due to age, poverty, or other factors. \\

    In order to run this algorithm, we first initialize a vector of county populations and matrix of transition probabilities. While the initial county populations are drawn from census data \cite{census}, justifications for the transition probabilities can be found in the next section. We standardize the time step for an update to be one hour such that multiplying the transition matrix and the current population vector gives the population distribution at the next hour. We do this until one of two termination criteria are met:
    \begin{itemize}
      \item[1.]
        The hurricane arrives after 96 time steps
      \item[2.]
        All population counts are below a predetermined, county-specific threshold
    \end{itemize}

    Once termination occurs, we evaluate the number of people evacuated in relation to quotas we determine using predicted landfall of a hurricane. This hurricane will be one randomly chosen from historical data. If population thresholds are not met within the full 4 day advance notice period, we suggest that the state be evacuated at first warning. Otherwise, evacuate in advance the number of hours required to meet the desired thresholds.

  \subsection{Parameter Values and Justification}
    With our earlier values for the maximum flow rate of an edge, we can now use them to derive transition probabilities for our Markov model. Because the maximum capacity of a road is 3000 times its weight in the graph, this can be used to give the total population emanating from a county in the first hour of the simulation. If we find the ratio of these values along each emanating edge to the initial county population, this gives us a good approximation of the percentage of a county that flees in any given time step. Again we assume these proportions are constant for each edge and ensure conservation of probability through the whole graph. We check conservation of probability by checking that the total population remains constant throughout the simulation.\\
    \\
    In order to get the threshold for each county based on the hurricane characteristics, we created a function that would give us the proportion of the original population that did not need to be evacuated. For instance, during a category 1 hurricane, we would find it acceptable to only evacuate 50\% of the original population. In addition to this, we required that the proximity of the county in question should factor into that threshold. The formula we ended up using is the following:
    \begin{align*}
        threshold(c, h) = 0.1 \cdot 2^{||c-h||_2 / h.radius} \cdot \frac{1}{h.category}
    \end{align*}
    Where c is the position of the county we are considering, h is the center of the predicted hurricane, and h.category is the category of the predicted hurricane. The relationship between these variables was chosen so that distance from the landfall would dominate the evacuation requirements, but they still get scaled by the category.This formula only applies for county coordinates within 3 hurricane radii of the landfall. If a county is further away than 3 times the radius of the hurricane, that county does not need to be evacuated, and we tag it as a refugee state (it essentially has infinite threshold). Furthermore, category 1 hurricanes have a threshold minimum of 50\% (we do not require more than half of the population of any county to evacuate). The heatmaps for those thresholds can be observed in Figure~\ref{fig:thresholds}. These were generated from a category 1 hurricane (a) that then grows in size (b) and finally becomes category 5 (c). Red nodes indicate counties where a majority of the population needs to be evacuated. Green nodes represent counties that are safe from significant hurricane damage:
    \begin{figure}[H]
      \center
      \begin{subfigure}[b]{0.3\textwidth}
        \center
        \includegraphics[width=\textwidth]{figures/right_73_1-crop.pdf}
        \caption{Radius: 73km - Category 1}
        \label{fig:pop_county60}
      \end{subfigure}~
      \begin{subfigure}[b]{0.3\textwidth}
        \center
        \includegraphics[width=\textwidth]{figures/right_120_1-crop.pdf}
        \caption{Radius: 120km - Category 1}
        \label{fig:pop_county39}
      \end{subfigure}~
      \begin{subfigure}[b]{0.3\textwidth}
        \center
        \includegraphics[width=\textwidth]{figures/right_120_5-crop.pdf}
        \caption{Radius: 120km - Category 5}
        \label{fig:pop_county60}
      \end{subfigure}
      \caption{Thresholds of evacuation for counties of Mississippi}
      \label{fig:thresholds}
    \end{figure}

  \subsection{Results}
    % TODO
      Figure~\ref{fig:pop_deter} shows the evolution of the populations of some example states during an evacuation. The Y-Axis shows the percentage of the original population is contained in that state, and the X-Axis is the number of hours that has elapsed since the evacuation order.
    \begin{figure}[H]
      \center
      \begin{subfigure}[b]{0.3\textwidth}
        \center
        \includegraphics[width=\textwidth]{figures/pop_county_29-crop.pdf}
        \caption{Population of Jackson}
        \label{fig:pop_county60}
      \end{subfigure}~
      \begin{subfigure}[b]{0.3\textwidth}
        \center
        \includegraphics[width=\textwidth]{figures/pop_county_39-crop.pdf}
        \caption{Population of Leake}
        \label{fig:pop_county39}
      \end{subfigure}~
      \begin{subfigure}[b]{0.3\textwidth}
        \center
        \includegraphics[width=\textwidth]{figures/pop_county_60-crop.pdf}
        \caption{Population of Rankin}
        \label{fig:pop_county60}
      \end{subfigure}
      \caption{Evolution of populations during evacuation}
      \label{fig:pop_deter}
    \end{figure}

    Running our test algorithm across all 15 hurricanes in our historical dataset, we found that 14 out of the 15 satisfied their threshold requirements within 80 hours of the evacuation start. Our recommendation for these hurricanes is to start the evacuation 80 hours before expected landfall. The one exception to this was a category 5 hurricane in the dataset. This hurricane took over 100 days to see the desired evacuation percentages, so we recommend an immediate and full evacuation.

  \subsection{Strengths and Weaknesses}
    A weakness of this model is that by aggregating population in the northernmost counties, we have a limiting assumption that evacuees never leave Mississippi. In reality they may want to flee to Arkansas or Alabama, rather then going all the way to the top of Mississippi. Also by having nonzero transition probabilities on every edge, citizens may flee northward when they are already in a safe zone.\\
    \\
    In ensuring that every highway fulfills its maximum theoretical utilization, this model assumes that evacuations are disorderly and don't follow any government mandated routes. Every citizen is told to simply flee, and they find the path of least resistance. This makes this model a good baseline for comparing more though fully orchestrated plans.\\
    \\
    Another strength of this model is that it intelligently decides evacuation priorities based on expected landfall. This allows for a more optimal solution to ensure we don't overreact to low class hurricanes, providing a minimal impact to the normal functioning of these counties.

\section{Maximum Flow Analysis of Strictly Northern Evacuations}
\label{sec:maxflow}
  After designing our first model, we realized that it was a naive assumption to have every road carrying traffic its maximum capacity. Actual flow rates would be dependent upon network structure as a whole and the resulting congestion. Because of this, we decided to perform a maximum flow analysis on our earlier strategy of evacuees moving northward.
  \subsection{Model Description}
    \par In this model, we will use the same approach of a network flow problem. The nodes of this graph will be the counties, and the edges are the roads connecting them together. Since we also have a quantifiable way to compare roads based on their potential throughput, we can label those as the capacities of the edges. In order to evaluate a south to north maximum flow, we add a super source node below our north-bound graph and connect it to every county on the coast using edges of infinite capacity. Similarly, we connect every county along the northern border of Mississippi to a super sink node using edges of infinite capacity. In order to get an idea of how many people can evacuate in a northbound manner, we can compute the maximum flow of this new network. The location of the dividing line between the two partitions will also tell us the areas of highest traffic congestion. This is because a maximum flow algorithm finds the minimum sized cut to partition a graph. Because all traffic entering the second partition must cross this cut, it represents a bottleneck for the flow.
  \subsection{Parameter Values and Justification}
    \par Now, in order to be able to compute the flow through the graph, we have to make it directed, and come up with a general rule for which directions we want to consider. After a lot of research, we saw that more often than not, going northbound was the best reaction to most hurricanes in Mississippi (because most of the hurricanes made landfall on the south coast). It is worth noting that we did not consider the interactions with the other states as that would require knowing a lot more things about the highway systems and tendencies of the evacuees in those states. We also assumed that the counties that would need to be evacuated are the ones on the southern border of Mississippi; we will call them the evacuated counties.
  \subsection{Results}
  \par We computeed the max flow using a min-cut algorithm that was implemented as part of the Python \texttt{graph-tool} library \cite{graphtool}. Since the units of road are all integers, we expected an integer as the maximum flow through the graph (South to North), and that value turned out to be 21000 people moving in that direction per hour. The partitions are shown in red and blue in Figure9 .
  \par The interesting part, though, is that this information tells us where the bottleneck is in terms of transportation. By inspection, we can see that a new road linking counties 20 and 56 (George to Perry, located on the south east corner of the state) could increase the maximum flow number, as well as push the congestion areas further north (which means people can get further away from the hurricanes faster).
  Indeed, after making this change, the max flow increased by over 25\% to 27000, and the partitions split much further north than before the addition of that road.
    \begin{figure}[H]
      \center
      \begin{subfigure}[b]{0.5\textwidth}
        \center
        \includegraphics[width=.5\textwidth]{figures/old_maxflow-crop.pdf}
        \caption*{Maximum Flow Partitions}
        \label{fig:oldmaxflow}
      \end{subfigure}~
      \begin{subfigure}[b]{0.5\textwidth}
        \center
        \includegraphics[width=.5\textwidth]{figures/maxflow_directed-crop.pdf}
        \caption*{Maximum Flow Partitions With George-Perry Connection}
        \label{fig:newmaxflow}
      \end{subfigure}
      \caption{}
    \end{figure}
    With this information, we can safely say that a great emergency evacuation plan could be prepared years in advance by creating new routes that link counties together. With the addition of a single two lane highway, we drastically increased the potential rate of evacuation of the southern counties. However, we can also use the maximum flow to determine how long in advance we need to order the evacuation.
    The total population of the evacuated counties is 616338. So let us compute the theoretical lower bound for how long it would take to evacuate those states fully:
    \begin{align*}
        t^* &= \frac{\text{total population}}{\text{max flow}}\\
        t^* &= \frac{532480}{21000}\\
        t^* &= 25 \text{hrs}
    \end{align*}
    So according to this model, if we decided to evacuate everyone in the southern counties all at once, it would take upwards of a day to get everyone to safe counties. Just to be safe, we can wait until two days prior to landfall before we order county wide evacuations. Because maximum flow algorithms can return the amount of flow traveling across every edge in the network, we can use this to set the traffic rates/evacuation routes for every highway connection. While this would require a great deal of government coordination, itts optimality guarantees

  \subsection{Strengths and Weaknesses}
    A great strength of this model is that it leverages an algorithm who's result is provably optimal. Rather than giving a general strategy to the governor, this allows us to specify the traffic that should be flowing across every single highway. Another strength is that it gives us a piece of information that no other model really touches: the roads that create a bottleneck in the evacuation.\\
    \\
    Potential downsides are that this model assumes a best case scenario for the utilization of the roads, whereas in reality, people will rarely use the fastest path to a destination. Their behavior will make for a much slower evacuation. Also, we don't really have a way to tell each car what route it should or shouldn't take in order to make the evacuation as speedy as possible, as it would mean sacrificing some drivers for the benefit of Pareto-optimality.

\section{Heuristic Model of Landfall-Avoiding Evacuations}
\label{sec:heuristic}
  \subsection{Model Description}
    \par In this model, instead of giving citizens particular evacuation plan, we give them a simple rule: move directly away from the hurricane landfall if possible, else move North or North-West. Using and assumed statewide broadcast system, the state of Mississippi can tell its citizens where the expected landfall will be, and which counties will be the most affected. In which case, individual citizens are instructed to leave the counties that will be flooded, and to do so as outlined above. This means less coordination from the state, and more options for each citizens. At the cost of not having designated areas, and therefore designated refuge areas for citizens, evacuation is faster because it involves less per-county organization.\\
    \par As opposed to the model described in \nameref{sec:maxflow}, we also allow citizens to leave the state of Mississippi. For example, Arkansas is less likely to be hit by the full force of hurricanes that move through Mississippi \cite{5news}, as such, it is a good place to take refuge while the storm passes. In order to achieve this, we add ``unofficial'' nodes in the graph representation of Mississippi. Figure~\ref{fig:highway_map} shows that Hancock and Jackson Counties have roads that lead into Louisiana and Alabama respectively, both of which are potential routes away from the Hurricane's landfall, meaning that such nodes are important when movement can also be lateral. Because we are only interested in the movement of citizens within Mississippi, we can assume that citizens that reach the out-of-state nodes keep moving further and further away from the hurricane, though we are not tracking them. We created four out-of-state nodes North, South, East and West of Mississippi, each connected in a way that satisfies the highways mapped in Figure~\ref{fig:highway_map}.
  \subsection{Parameter Values and Justification}
    \par The parameters to this model are the same as the parameters for the models outlined above, except that this model uses the complete highway map of Mississippi, instead of only considering vertical highways. This algorithm expects three inputs outside of the parameters: the first, less accurate estimate of the affected counties, the more accurate second estimate and the final, most accurate estimate. Notice that as described, these inputs actually abstract away the hurricane's category. Instead, the category is reflected by the size of the list of the affected counties given as the respective inputs for the estimates. In order to generate these input, we used the historical hurricane data. Given the landfall and the hurricane's category, we can generate three sets of decreasing size containing neighboring counties. We can then run the simulation on these inputs. For the first 48 hours, we start to evacuate all the counties listed in the less accurate estimated, then for the second to last day we use the more accurate set of counties to pick which ones to evacuate, and finally the most accurate estimate is used for the last 24 hours.
  \subsection{Results}
    \par This model's main goal is to find the amount of time it takes to evacuate the affected counties, given the estimated floods. After 100 simulations with the parameters described above, we find that it consistently takes less than 50 hours to evacuate all the at-risk counties. Figures \ref{fig:stochastic_out_large} and \ref{fig:stochastic_out_small} depict the evolution of the county populations when the impending hurricane is of category 5 and 1 respectively. The hurricane in Figure~\ref{fig:stochastic_out_large} is coming from the West, and a such, the population of Jackson County initially peaks as everyone moves east and directly away from the hurricane's projected landfall, then decreases as citizens start then making their way north and out of the state. In Figure~\ref{fig:stochastic_out_small}, the hurricane comes directly from the South. There is a slight flaw in the graph in that Harrison County is represented to be lower than Jackson County, and the connectivity between the two is quite high. As such, people from Harrison County will seek refuge in Jackson County before moving North.
    \par There was also a large congestion observed in Greene County. It would seem that on their way North, citizens have trouble exiting this county. The model described in \nameref{sec:maxflow} concluded that adding highways in this area of Mississippi would drastically improve the flow of citizens. This heuristic model reflects that this is a high congestion area and supports the idea that adding roads there would increase citizen flow.
    \begin{figure}[H]
      \center
      \begin{subfigure}[b]{0.5\textwidth}
        \center
        \includegraphics[width=\linewidth]{figures/pred_hancock-crop.pdf}
        \caption{Predicted Evacuation for Landfall in Hancock County}
        \label{fig:stochastic_out_large}
      \end{subfigure}~
      \begin{subfigure}[b]{0.5\textwidth}
        \center
        \includegraphics[width=\linewidth]{figures/pred_jackson-crop.pdf}
        \caption{Predicted Evacuation for Landfall in Jackson County}
        \label{fig:stochastic_out_small}
      \end{subfigure}
      \caption{Evacuation Comparison for Category 5 and 1 Hurricanes}
      \label{fig:stochastic_out}
    \end{figure}
  \subsection{Strengths and Weaknesses}
    \par There are three main weaknesses in this model, two of which are referred to above. The first is that this model assumes that the State of Mississippi does not make accommodations for its citizens in the northern counties, meaning that once the citizens evacuate their home counties, they are left to fend for themselves. But, it does mean that evacuation is very fast, since no county officials need to be involved in organizing the evacuation. There comes a point at which the State must leverage speed of evacuation against citizen accommodations. We can only help in stating that a strategy such as the one outlined in this model will likely yield quicker evacuation times, especially in times of emergency.
    \par The other weakness is that by representing Mississippi like we have done, we prevent citizens from using smaller backroads to leave the county. This hinders their heuristic to move NorthWest in our simulations. Had we used a more accurate dataset or a more fine grid than just county geographical centers, the simulations would yield faster evacuation times, with everyone immediately going in the right direction.

\section{Conclusions}
\label{sec:conclusions}
\par The first thing that we want to conclude on is that two of our models agree on the fact that there is a large amount of congestion in the Southern area of Mississippi. Congestion in this area is not ideal, and in fact dangerous, since it is the area most likely to be flooded by hurricanes. This congestion seems to be caused by a lack of highway coverage in the Southwestern part of Mississippi, especially between Stone, Perry, Greene and George counties. In fact, adding a Four-Lane Highway between George and Perry Counties moves the bottleneck of traffic flow to a considerably safer area up North. Our strongest recommendation to the State of Mississippi, as  a preventative measure, is to consider such a highway.\\

\par Finally, we can conclude on the strategy to use, depending on the situation. The Heuristic Model is a model where all citizens decide where they are going on their own, and in which county officials do not need to intervene. As such, this model leads to the highest evacuation speeds, at the cost of having a place for the citizens to take refuge. Therefore this strategy should only be used in the most urgent scenario, where everyone absolutely must leave the affected counties as quickly as possible. On the other hand, if the State has had more time to prepare and has forecasted the arrival of the hurricane early enough, then they can tell the citizens to stay in-state, make accommodations for them, and give them very specific routes according to either the Markov Model or the Maxflow Model. This is a more heavy-handed strategy and would invariably involve a lot of intervention from the county officials in order to properly organize the evacuations.

\section{Future Work}
\label{sec:future}
The first aspect that we would like to focus on in the future had we had more time would be the interactions between Mississippi and its surrounding states. It is probable that in the event of a state wide evacuation, the transportation system would need to support not just the citizens of Mississippi but also those of Louisiana and Alabama.
\newline
In addition to this, we would like to do more robustness testing by generating random hurricanes and their trajectories, and plan accordingly. This would allow us to test on a much bigger data sample than the one we were able to get our hands on.
\newline
Another area that we would have liked to improve has to do with the fact that we did not take into account the trajectory of the hurricanes after they made landfall. We considered that by that time, the populations of the counties affected should have been evacuated according to the level of emergency that we assigned them.\newline
Overall, all of these would be achievable with the state of our models, but would require a lot of time to polish as it would add a significant amount of complexity to the simulations.

\section{Individual Contributions}
\label{sec:contributions}
\subsection*{Ryan}
Performed analysis of historical hurricane data. Contributed to writing and revision of the manuscript. Helped to work out the design and logic for both the MaxFlow and the Markov model.

\subsection*{Paul}
Designed and implemented the stochastic model. Helped Ryan write the manuscript. Parsed the county data such as county location and population using the WolframAlpha API. Created the Mississippi Highway map used in all the models.

\subsection*{Antoine}
Implemented the MaxFlow and Markov models. Created almost all of the graph and network visualizations. Worked on the two first models' analyses for the manuscript. Wrote the Future Work section and the Individual Contributions.

  \begin{thebibliography}{9}
    \bibitem{county_map}
      \url{http://www.madeinmississippi.us/wp-content/uploads/2015/02/mississippi-county-map.jpg}
    \bibitem{ms_highways}
      \url{https://www.mississippi.org/assets/docs/library/ms_highways.pdf}
    \bibitem{chaos}
      \url{http://www.mapcruzin.com/images/hurricane-shapefile-400x255.png}
    \bibitem{smaller_chaos}
      \url{https://coast.noaa.gov/hurricanes/}
    \bibitem{fifteen}
      \url{http://www.aoml.noaa.gov/hrd/hurdat/Data_Storm.html}
    \bibitem{categories}
      \url{http://www.nhc.noaa.gov/aboutsshws.php}


    \bibitem{pmc}
      \url{http://www.ncbi.nlm.nih.gov/pmc/articles/PMC4060166/}
    \bibitem{5news}
      \url{http://5newsonline.com/2012/08/27/garretts-blog-hurricanes-in-arkansas/}
    \bibitem{famsize}
      \url{http://www.statista.com/statistics/183657/average-size-of-a-family-in-the-us/}
    \bibitem{graphtool}
      \url{http://graph-tool.skewed.de/}
    \bibitem{census}
      \url{http://census.ire.org/data/bulkdata.html}
  \end{thebibliography}

\lstinputlisting[style=Matlab]{deterministic/certainties.m}
\lstinputlisting[style=Matlab]{deterministic/constantrate.m}
\lstinputlisting[style=Matlab]{deterministic/getA.m}
\lstinputlisting[style=Matlab]{deterministic/populations.m}
\lstinputlisting[style=Matlab]{deterministic/thresholds.m}
\lstinputlisting[style=Matlab]{hurricane_generator/interpolate_radius.m}
\lstinputlisting[style=Python]{deterministic/threshold.py}
\lstinputlisting[style=Python]{get_pop.py}
\lstinputlisting[style=Python]{stochastic/data.py}
\lstinputlisting[style=Python]{stochastic/simulation.py}
\lstinputlisting[style=Python]{stochastic/wolfram.py}
\lstinputlisting[style=Python]{get_adjacency.py}
\lstinputlisting[style=Python]{utils.py}
\lstinputlisting[style=Python]{hurricane_generator/clean.py}
\lstinputlisting[style=Python]{hurricane_generator/get_radii.py}
\lstinputlisting[style=Python]{hurricane_generator/geotag.py}
\lstinputlisting[style=Python]{hurricane_generator/final.py}
\lstinputlisting[style=Python]{flow/colornodes.py}
\lstinputlisting[style=Python]{flow/maxflow.py}
\lstinputlisting[style=Python]{flow/create_graph.py}
\lstinputlisting[style=Python]{flow/threshold_from_hurricane.py}

\end{document}